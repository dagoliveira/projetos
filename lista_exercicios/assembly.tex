\documentclass{article}

\usepackage[brazilian]{babel}
\usepackage[utf8]{inputenc} % arruma utf8
\usepackage[T1]{fontenc} % arruma ctrl+c com utf8
\usepackage{indentfirst}

\title{Lista de Exercícios RISC-V}
\author{Gabriel G. de Brito \and Daniel Oliveira}
\date{Segundo semestre de 2024}

\begin{document}

\maketitle

A melhor maneira de resolver esses exercícios é com o debugger do emulador EGG.
Lembre sempre da instrução \textit{ebreak} no final!

\tableofcontents

\section{Instruções Aritméticas}

Para testar suas resoluções aos exercícios dessa seção, carregue seu código no
debugger do emulador e use os comandos \textit{set} e \textit{print} para
modificar manualmente o valor dos registradores e verificá-los depois.

Por exemplo, para testar a resolução a um exercício hipotético que deve somar 1
ao valor em \textit{a0}:

\begin{verbatim}
$ egg -d teste.asm
EGG - Emulador Genérico do Gabriel - versão 3.0.0
Entre 'help' para ver uma lista de comandos
Debuggando RISC-V IM (32 bits)
egg> set a0 41
egg> continue
Parado na chamada BREAK no endereço 0x8
nenhuma instrução no endereço 0x8
Registrador a0: mudou de 0x00000000 para 0x0000002a
egg> print a0
0x0000002a
egg> q
até mais!
\end{verbatim}

% Eu tentei usar o pacote exercise porém apesar da doc dizer que ele suporta
% babel, eu não consegui escrever em português.
\begin{enumerate}
  \item Escreva um programa que calcule a média entre dois números armazenados
  nos registradores \textit{a0} e \textit{a1}. Guarde o resultado em
  \textit{a0}.

  \item Resolva o exercício anterior sem utilizar instruções de multiplicação.

  \item Escreva um programa que calcule a média entre três números armazenados
  nos registradores \textit{a0}, \textit{a1} e \textit{a2}. Guarde o resultado
  em \textit{a0}.

  \item Escreva um programa que calcule a expressão $b^2 - 4 \cdot a \cdot c$ (o
  $\Delta$ de uma equação de segundo grau), onde \textit{a}, \textit{b} e
  \textit{c} estão armazenados nos registradores \textit{a0}, \textit{a1} e
  \textit{a2}, respectivamente. Guarde o valor em \textit{a0}.

  \item Resolva o exercício anterior utilizando somente duas instruções de
  multiplicação.

  \item Escreva um programa que, dados números \textit{a} e \textit{b}
  armazenados nos registradores \textit{a0} e \textit{a1}, respectivamente,
  guarde o valor 1 em \textit{a0} caso $a < b$, e o valor 0 caso contrário. Não
  use instruções de salto, \textit{slt} ou \textit{sltu}.

  \item Escreva um programa que, dado um número armazenado em \textit{a0},
  guarde em \textit{a0} o valor do segundo byte desse número e em \textit{a1} o
  valor do quarto byte. Exemplo de execução:

  \begin{verbatim}
$ egg -d solucoes/bytes.asm
EGG - Emulador Genérico do Gabriel - versão 3.0.0
Entre 'help' para ver uma lista de comandos
Debuggando RISC-V IM (32 bits)
egg> set a0 0xcafebabe
egg> c
Parado na chamada BREAK no endereço 0x18
nenhuma instrução no endereço 0x18
Registrador t0: mudou de 0x00000000 para 0x000000ba
Registrador a0: mudou de 0x00000000 para 0x000000ba
Registrador a1: mudou de 0x00000000 para 0x000000ca
egg> p a0
0x000000ba
egg> p a1
0x000000ca
egg> q
até mais!
  \end{verbatim}

  \item Resolva o exercício anterior sem utilizar as instruções \textit{and} e
  \textit{andi}. Dica: use \textit{xor} e \textit{slli}.

  \item Escreva um programa que calcule o valor em representação sinal-magnitude
  do número em complemento de dois armazenado no registrador \textit{a0}. Guarde
  o resultado em \textit{a0}. Não utilize instruções de salto. Nota: números
  negativos em sinal magnitude vão aparecer com os últimos dígitos hexadecimais
  8, 9, A, B, C, D, E ou F. Exemplo: -1 é 0x80000001.

  \item (Desafio) Computadores modernos costumam suportar operações aritméticas
  nativas de números de 32 ou 64 bits. Caso seja necessário fazer aritmética com
  números maiores, é necessário implementar funções em software que utilizam as
  instruções da arquitetura para somar números maiores - de 128 bits por
  exemplo. Essas funções costumam ser chamadas de operações em ``BigInt'' e
  demoram mais, pois, ao contrário de operações em números suportados pela
  arquitetura, precisam de muito mais que apenas uma instrução para serem
  executadas.

  A arquitetura que estudamos - RISC-V 32 IM - suporta apenas números de até 32
  bits. Escreva um programa que some dois números de 64 bits A e B, onde a parte
  baixa de A está armazenada em \textit{a0}, a parte alta em \textit{a1} e,
  semelhantemente, a parte baixa de B está armazenada em \textit{a2} e a parte
  alta em \textit{a3}. Não utilize instruções de salto. Guarde a parte baixa do
  resultado em \textit{a0} e a parte alta em \textit{a1}.
\end{enumerate}

\section{Instruções de salto}

Semelhantemente à seção anterior, utilize os comandos \textit{set} e
\textit{print} para testar as soluções.

\begin{enumerate}
  \item Escreva um programa que calcule o fatorial de um número armazenado no
  registrador \textit{a0}, guardando o resultado em \textit{a0}.

  \item Escreva um programa que calcule o n-ésimo número da sequência de
  Fibonacci, onde n está no registrador \textit{a0}, armazenando-o no
  registrador \textit{a0}. Note que tanto o primeiro quanto o segundo número de
  Fibonacci são 1.

  \item Escreva um programa que calcule a quantidade de bits 1 em um número
  binário armazenado no registrador \textit{a0}. Guarde o resultado em
  \textit{a0}.

  \item Escreva um programa que conte a quantidade de ``trailing zeros'' de um
  número binário armazenado no registrador \textit{a0}, guardando o resultado em
  \textit{a0}. ``Trailing zeros'' são os zeros à direita do número, ou seja, os
  bits zero antes do primeiro bit 1. Por exemplo, o número de ``trailing zeros''
  do número binário 1011000 é 3.

  \item Escreva um programa que verifique se um número armazenado em \textit{a0}
  é primo. Guarde 1 em \textit{a0} caso seja, e 0 caso contrário.

  \item Escreva algoritmos (em prosa ou pseudocódigo) para transformar os 3
  tipos de loop da linguagem C (while, for e do-while) em Assembly.

  \item (Desafio) Escreva um programa que faça um loop 5 vezes, porém a única
  instrução de salto permitida é \textit{jalr}, e não é permitido o uso do seu
  imediato (ou seja, o imediato de todas as ocorrências de \textit{jalr} no
  código deve ser 0). Chamadas de ambiente (\textit{ecall} e \textit{ebreak})
  também não são permitidas. Dica: utilize as instruções \textit{slti} e
  \textit{andi}.
\end{enumerate}

\section{Instruções de acesso à memória}

Semelhantemente as seções anteriores, utilize os comandos \textit{set} e
\textit{print} para testar as resoluções, dessa vez também na memória (com a
sintaxe @). Crie também vetores no próprio código.

Nessa seção, é de extrema importância lembrar que inteiros são armazenados em 4
bytes na memória de processadores RISC-V!

Alguns exercícios dessa seção exigem conhecimento sobre a tabela ASCII.

\begin{enumerate}
  \item Escreva um programa que armazene os n primeiros números naturais em um
  vetor, onde n está armazenado no registrador \textit{a0}.

  \item Escreva um programa que some os elementos de dois vetores A e B na
  memória, guardando em um terceiro vetor C, tal que $C_i = A_i + B_i$.

  \item Escreva um programa que armazene a sequência de Fibonacci até o n-ésimo
  número na memória, onde n está armazenado no registrador \textit{a0}. Para
  simplificar, considere apenas $n \geq 2$.

  \item Escreva um programa que calcule a média aritmética dos valores em um
  vetor e guarde o resultado no registrador \textit{a0}.

  \item Dados dois vetores de mesmo tamanho: um vetor de valores e outro de
  pesos, escreva um programa que calcule a média ponderada dos valores do
  primeiro vetor com seus respectivos pesos no segundo vetor. Guarde o resultado
  no registrador \textit{a0}.

  \item (ASCII) Escreva um programa que transforme um texto de letras minúsculas
  para maiúsculas, utilizando um loop com instruções de salto. Use
  \textit{ecall} para testar. Exemplo:

  \begin{verbatim}
    addi a7, zero, 3
    addi a0, zero, texto
    addi a1, zero, 13 ; tamanho do texto
    ecall

    ; aqui vai o código para transformar o texto.

    addi a7, zero, 3
    addi a0, zero, texto
    addi a1, zero, 13 ; tamanho do texto
    ecall

    ebreak

  texto:
  #isso eh texto
  \end{verbatim}

\end{enumerate}

\section{Chamada de funções}

Siga a convenção de registradores e chamada de função SEMPRE!

\begin{enumerate}
  \item Traduza o seguinte programa de C para Assembly. Use \textit{ecall} para
  traduzir as chamadas \textit{printf}. Considere que o argumento ``a0'' da
  função main é o registrador \textit{a0}, use o comando \textit{set} do
  emulador para testar o programa com diferentes números.

  \begin{verbatim}
    int fatorial(int n) {
      int f = 1;
      for (int i = 2; i <= n; i++) {
        f = f * i;
      }
      return f;
    }

    int main(int a0) {
      f = fatorial(a0);
      if (f > a0 * 10) {
        printf("O fatorial eh mais que 10x maior.\n");
      } else {
        printf("O fatorial eh menos que 10x maior.\n");
      }
    }
  \end{verbatim}

  \item Resolva o exercício anterior, porém troque a implementação da função de
  fatorial por uma recursão.
\end{enumerate}

\end{document}
